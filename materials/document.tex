\documentclass[12pt]{article}
\usepackage{amsmath}
\usepackage{amsfonts}
\usepackage{amssymb}
\usepackage{geometry}
\geometry{a4paper, margin=1in}

\title{Lecture 4: Discrete Random Variables and the Binomial Lattice Model}
\author{}
\date{}

\begin{document}
	
	\maketitle
	
	\section*{Introduction to the Binomial Lattice Model}
	\begin{itemize}
		\item \textbf{Conceptual Foundation:}
		\begin{itemize}
			\item Define the binomial lattice model as a discrete-time model for asset price dynamics.
			\item Recap Bernoulli and binomial random variables and connect them to price movements: up and down steps driven by Bernoulli trials.
		\end{itemize}
		\item \textbf{Mathematical Framework:}
		\begin{itemize}
			\item Asset price $S_t$ at time $t$ evolves as:
			\[
			S_{t+1} = 
			\begin{cases} 
				u S_t & \text{with probability } p, \\
				d S_t & \text{with probability } 1-p,
			\end{cases}
			\]
			where $u > 1$, $d < 1$, and $p$ is the probability of an upward movement.
		\end{itemize}
	\end{itemize}
	
	\section*{Calibrating Parameters for Small Time Steps}
	\begin{itemize}
		\item \textbf{Role of Calibration:}
		\begin{itemize}
			\item Calibrate $u$, $d$, and $p$ to ensure consistency with the expected return and volatility for small time intervals ($\Delta t$).
		\end{itemize}
		\item \textbf{Approximation for Small $\Delta t$:}
		\[
		u = e^{\sigma \sqrt{\Delta t}}, \quad d = e^{-\sigma \sqrt{\Delta t}}, \quad p = \frac{e^{r \Delta t} - d}{u - d},
		\]
		where $\sigma$ is volatility and $r$ is the risk-free rate.
		\item Highlight the intuitive link between the discrete-time model and continuous models (to be expanded in Lecture 5).
	\end{itemize}
	
	\section*{Implementing the Model in R}
	\begin{itemize}
		\item \textbf{Programming Concepts:}
		\begin{itemize}
			\item Start by implementing a simple binomial tree in R:
			\begin{itemize}
				\item Generate the tree structure (price paths) iteratively or recursively.
				\item Compute probabilities along paths to calculate option values or expected prices.
			\end{itemize}
			\item Use this example to introduce:
			\begin{itemize}
				\item \textbf{Control and Flow Structures:}
				\begin{itemize}
					\item \texttt{if} statements for branching logic.
					\item Loops (\texttt{for}, \texttt{while}, \texttt{repeat}) for iterative computations.
				\end{itemize}
				\item \textbf{Modularization:}
				\begin{itemize}
					\item Break the program into smaller functions:
					\begin{enumerate}
						\item A function to compute $u$, $d$, and $p$.
						\item A function to generate the binomial tree.
						\item A function to calculate prices or option values.
					\end{enumerate}
					\item Combine functions into a cohesive program.
				\end{itemize}
				\item \textbf{Lists in R:}
				\begin{itemize}
					\item Use lists to store and organize data such as the levels of the binomial tree, probabilities, and computed prices.
				\end{itemize}
